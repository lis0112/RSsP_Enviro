%\chapter{Zadání}
\chapter{Zadání} 
\label{sec:Zadani}
\section{Specifikace parkoviště}
Ovládání světel lampy na základě různých enviro podmínek
Aplikace bude měřit nějaké rozumné veličiny (tmu, déšť, vítr, smog) a z nich se
vytvoří požadavek na přiměřené osvětlení parkoviště. Toto bude zasláno rozhodující
aplikaci, jako v předchozím případě

\section{Specifikace dokumentace}
\begin{enumerate}[label=\arabic*)]
    \item Analýza technologického řešení
    
    Na základě zadání je potřeba analyzovat hardware dané technologie, což znamená:
    \begin{itemize}
        \item volbu typu systému – distribuovaný, centralizovaný – třeba neopomenout důvod;
        \item volbu senzorů;
        \item zapojení – komunikace, řídících jednotek, silového vedení, aktuátorů, senzorů;
        \item volbu silových jednotek, aktuátorů;
        \item volbu vizualizační prostředí;
        \item sledování a ukládání dat;
        \item a jiné.
    \end{itemize}
    Výsledkem je sada výkresů subcelků, seznamy komponent, dokumentace komponent, zdroje
informací.
    \item Dokumentace technologie
    
    Zde je očekáván výstup ve formě výkresu celé technologie (případně její části). Cílem je zachytit podstatu celku a jeho částí, najít případné mezery v rámci komunikace.

    Poznámka: K tvorbě výkresů lze využít vektorové programy (Autocad, CorelDraw, atd.), tužku s papírem a scannerem či jiné projektové aplikace.

    \item Softwarová analýza
    
    Součástí softwarové analýzy je:
    \begin{itemize}
        \item Obecná analýza
        \begin{itemize}[\tiny \ding{109}]
            \item slovní forma
        \end{itemize}
        \item Analýza struktury vnějšího prostředí
        \begin{itemize}[\tiny \ding{109}]
            \item interakce lidí se softwarem – kdo a jak může se softwarem pracovat
        \end{itemize}
        \item Analýza funkcí
        \begin{itemize}[\tiny \ding{109}]
            \item funkce, které aplikace umožňuje
            \item provádění funkcí – kdy a jak často se mají provádět
        \end{itemize}
        \item Analýza komunikací
        \begin{itemize}[\tiny \ding{109}]
            \item komunikace mezi jednotlivými částmi aplikace – jak komunikuje hlavní řídicí algoritmus
            s ostatními částmi aplikace
        \end{itemize}
        \item Analýza dokumentů
        \begin{itemize}[\tiny \ding{109}]
            \item všechny dokumenty, které jsou generovány nebo používány v aplikaci – co budou dané
            dokumenty obsahovat
        \end{itemize}
        \item Analýza obsahu a struktury informací
        \begin{itemize}[\tiny \ding{109}]
            \item typ a struktura dat v systému
            \item frekvence zpracování a používané přenosy dat
            \item délka uchovávání dat
        \end{itemize}
        \item Analýza toku informací
        \begin{itemize}[\tiny \ding{109}]
            \item toky dat mezi jednotlivými funkcemi
            \item ochrana dat
        \end{itemize}
        \item Analýza slabých míst
        \begin{itemize}[\tiny \ding{109}]
            \item identifikace problémů, opomenutí a redundancí (funkcí i celého systému)
        \end{itemize}
    \end{itemize}

    \item Systémová specifikace
    
    \begin{itemize}
        \item Výchozí situace a cíle
        \begin{itemize}[\tiny \ding{109}]
            \item cíle a účel softwaru
            \item aktuální funkcionalita – co lze nabídnout zákazníkovi
        \end{itemize}

        \item Vztah okolí k provozování systému
        \begin{itemize}[\tiny \ding{109}]
            \item podmínky pro provoz
            \item jaká vnější data jsou potřeba
            \item počet uživatelů, jejich činnosti, frekvence užití
        \end{itemize}

        \item Funkční požadavky
        \begin{itemize}[\tiny \ding{109}]
            \item seznam funkcí softwaru očekávané uživatelem – co očekáváme od technologie vzhledem k softwaru
        \end{itemize}

        \item Nefunkční (ostatní) požadavky
        \begin{itemize}[\tiny \ding{109}]
            \item požadavky na spolehlivost, přenositelnost
            \item reakční časy a doba zpracování
        \end{itemize}

        \item Uživatelská rozhraní
        \begin{itemize}[\tiny \ding{109}]
            \item popis nedůležitějších bodů uživatelského rozhraní
            \item popisuje způsob a prostředky, jimiž uživatel komunikuje se systémem
        \end{itemize}

        \item Chování za chybových situací
        \begin{itemize}[\tiny \ding{109}]
            \item rozbor vlivů různých chyb a požadované chovaní systému při jejich výskytu
        \end{itemize}

        \item Požadavky na dokumentaci
        \begin{itemize}[\tiny \ding{109}]
            \item referenční příručka, manuál, systémová dokumentace
        \end{itemize}

        \item Předávací podmínky
        \begin{itemize}[\tiny \ding{109}]
            \item návrh testů a způsobu kontroly pro každý požadavek samostatně
        \end{itemize}

        \item Přílohy
        \begin{itemize}[\tiny \ding{109}]
            \item pojmy, bibliografie atd.
        \end{itemize}
    \end{itemize}

    \item UML analýza
    
    Analýza pomocí UML diagramů bude obsahovat minimálně tolik diagramů UML, kolik je studentů ve skupině (například: diagram užití, aktivitní diagram, diagram tříd, stavový diagram, sekvenční diagram, časování). Každý ze studentů tedy vytvoří alespoň jeden z těchto diagramů.

    \item Výstupy projektu
    
    Jsou očekávány dva výstupy, jež budou uloženy pomocí GIT ve vzdáleném repositáři včetně všech dodatečných souborů a příloh. Výstupy jsou:
    \begin{itemize}
        \item dokument splňující veškeré body zadání 1 až 5 a obsahující titulní list, obsah, patřičné
        formátování, schémata, obrázky, diagramy, přílohy a self-assessment (sebehodnocení přínosu
        jednotlivých členů týmu – kdo co udělal);
        \item prezentace výsledku projektu (28.11.2022).   
    \end{itemize}

\end{enumerate} 



\endinput